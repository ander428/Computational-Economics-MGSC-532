\documentclass{article}
\usepackage{graphicx}
\usepackage{amsmath}

\begin{document}

\title{%
  Probability of a Flush \\
  \large Computational Economics - MGSC 532}
\author{Joshua Anderson}

\maketitle

\begin{*}
    Here is the calculation of the probability of a flush in
    the game of poker:
\end{*}

\begin{equation}
    P(every flush) = 
    \dfrac{
        \begin{pmatrix}13 \\ 5\end{pmatrix}
        \begin{pmatrix}4 \\ 1\end{pmatrix}
    }{
        \begin{pmatrix}52 \\ 5\end{pmatrix}
    }
    =
    \dfrac{
        \dfrac{13!}{5!(13-5)!}\times\dfrac{4!}{1!(4-1)!}
    }{
        \dfrac{52!}{5!(52-5)!}
    }
    =
    \dfrac{1287\times4}{2598960}
    =
    0.00198
\end{equation}

\begin{equation}
    P(straight flush) = 
    \dfrac{
        \begin{pmatrix}9 \\ 1\end{pmatrix}
        \begin{pmatrix}4 \\ 1\end{pmatrix}
    }{
        \begin{pmatrix}52 \\ 5\end{pmatrix}
    }
    =
    \dfrac{
        \dfrac{9!}{1!(9-1)!}\times\dfrac{4!}{1!(4-1)!}
    }{
        \dfrac{52!}{5!(52-5)!}
    }
    =
    \dfrac{9\times4}{2598960}
    =
    1.385e^{-5}
\end{equation}

\begin{equation}
    P(royal flush) = 
    \dfrac{
        \begin{pmatrix}4 \\ 1\end{pmatrix}
    }{
        \begin{pmatrix}52 \\ 5\end{pmatrix}
    }
    =
    \dfrac{\dfrac{4!}{1!(4-1)!}}{\dfrac{52!}{5!(52-5)!}}
    =
    \dfrac{4}{2598960}
    =
    1.539e^{-6}
\end{equation}

\begin{equation}
    P(flush) = P(every flush) - P(straight flush) - P(royal flush)
\end{equation}

\begin{equation}
    P(flush) = 0.00198 - 1.385e^{-5} - 1.539e^{-6} = 0.001964611
\end{equation}
\newline

\begin{center}
The probability of a flush in a hand of poker is about 0.196\%.
\end{center}
\end{document}